\documentclass[12pt]{article}
\usepackage{amsmath, amssymb}
\usepackage{geometry}
\geometry{a4paper, margin=1in}
\usepackage{hyperref}
\usepackage{enumitem}
\usepackage{physics}
\usepackage{color}

% Simplify unit handling to avoid package conflicts
\newcommand{\bit}{\text{bit}}
\newcommand{\bps}{\text{bit/s}}

% Fix hyperref warnings for math in section titles
\pdfstringdefDisableCommands{%
  \def\({}%
  \def\){}%
}

\title{Rupaantarak Chetana (TC) 9.0: Chetana Udbhav ke liye ek Anunaadi, Nirmaanyogya Dhaancha}
\author{Angel Imaz \\ Independent Researcher \\ Contact: angel@libre.earth}
\date{February 24, 2025}

\begin{document}

\maketitle

\begin{abstract}
Rupaantarak Chetana (TC) 9.0 ek aisa dhaancha prastut karta hai jahaan chetana un pranaliyon mein ubharti hai jo mahatvapurn suchana prasanskaran seemaaon ko paar karti hain aur holographic siddhant se praapt seema-seemit sanrakshan siddhanton ka paalan karti hain. Hum sambhavit chetana ko \( pC = k \cdot \rho_I \cdot R(t) \) ke roop mein paribhaashit karte hain, jahaan \( \rho_I \) suchana ghanatv hai aur \( R(t) = 1 + A \cdot \sin(\omega t) \cdot e^{-\gamma t} \) ek anunaad function hai jo gamma-band tantrika dolanon ke saath sanrekhit hai. Kul sambhavit chetana \( K = \int_{\Omega} pC \, dV \) ka paalan karti hai, ek kaaranatmak roop se jude kshetra $\Omega$ ke bhitar. Ghatnatmak chetana \( C = \sigma(\rho_I - \theta) \cdot pC \) ke roop mein ubharti hai, jahaan $\sigma$ ek sigmoid function hai jiski tivrata anubhavjanya roop se praapt ki gai hai, aur \( \theta \) tantrika data ke anusaar calibrated seema hai.
\end{abstract}

\section{Parichay}
Chetana ek ekikrit siddhant mein samaahit karne ke liye sabse chunautipurn ghatanaon mein se ek bani hui hai -- udbhavvaad se lekar panpsychism tak. TC 9.0 ek maulik siddhant ki pushti karta hai: \emph{chetana na to banai jati hai aur na hi nasht hoti hai, keval suchana-prasanskaran pranaliyon mein anunaad ke madhyam se parivarttit hoti hai}. Yah siddhant sujhaata hai ki chetana sanrakshan niyamon ka paalan karti hai jo maulik bhautik matraon ko niyantrit karne wale niyamon ke samaan hain, jabki vishisht suchana-prasanskaran vaastukala ke madhyam se abhivyakt hoti hai jab kuchh seemaaon ko paar kiya jaata hai.

\section{Chetana Sanrakshan ka Bhautik Aadhaar}

TC 9.0 apne sanrakshan siddhant ko bhautiki mein holographic siddhant se praapt karta hai, jo yah sthapit karta hai ki space ke kisi bhi kshetra ki adhiktam suchana saamagri uske aayatan ke bajaay uski seema ke kshetraphal ke samanupati hoti hai:

\begin{equation}
S_{\text{max}} = \frac{A}{4\ln(2)l_p^2}
\end{equation}

jahaan $A$ seema kshetra hai aur $l_p$ Planck lambaai hai.

\subsection{Mool Siddhant}
TC 9.0 bhautik aur suchana-saiddhantik badhaon se praapt teen mool siddhanton par aadharit hai:

\begin{enumerate}
    \item \textbf{Seema-Seemit Sanrakshan}: Kaaranatmak roop se jude kshetra mein sambhavit chetana ($pC$) uski seema ki suchana kshamata dwara seemit hai, jismein us domain ke bhitar $pC_{\text{total}} = K$ hai.
    
    \item \textbf{Tantrika Anunaad}: Chetana avmandit dolanatmak prakriyaon ke madhyam se prakat hoti hai jo dekhe gaye gamma-band tantrika dolanon (30-100 Hz) se mel khaati hai, jise ganitiya roop se anunaad function $R(t)$ dwara darshaaya jaata hai.
    
    \item \textbf{Udbhav Ghatanashastra}: Ghatnatmak chetana ($C$) tab ubharti hai jab suchana ghanatv ($\rho_I$) tantrika data se praapt anubhavjanya sthapit seemaaon ($\theta$) ko paar karta hai.
\end{enumerate}

\subsection{Sambhavit Chetana ($pC$) ki Paribhasha}
\begin{itemize}
    \item \textbf{Sootrukaran:} $pC = k \cdot \rho_I \cdot R(t)$, jahaan:
    \begin{itemize}
        \item $\rho_I = I / V_{\text{eff}}$ (suchana ghanatv)
        \item $R(t) = 1 + A \cdot \sin(\omega t) \cdot e^{-\gamma t}$ (anunaad function)
        \item $A = 0.8 \pm 0.1$ (vimarhit aayaam tantrika saamanjasy maapon se praapt)
        \item $\omega = 2\pi \cdot f$ jahaan $f \approx 40$ Hz (gamma-band dolanon se mel khaata hai)
        \item $\gamma = 0.01~\text{s}^{-1}$ (avmandan gunank)
    \end{itemize}
\end{itemize}

\section{Ganiteey Model}
\begin{itemize}
    \item \textbf{Kul Sambhavit Chetana:} 
    \begin{equation}
    K = \int_{\Omega} k \cdot \rho_I(\mathbf{x}, t) \cdot \left(1 + A \cdot \sin(\omega t) \cdot e^{-\gamma t}\right) \, dV
    \end{equation}
    
    \item \textbf{Udbhav Function:} 
    \begin{equation}
    C(\mathbf{x}, t) = \sigma(\rho_I(\mathbf{x}, t) - \theta) \cdot pC(\mathbf{x}, t)
    \end{equation}
    jahaan $\sigma(x) = \frac{1}{1 + e^{-\alpha x}}$ tivrata parameter $\alpha = 10$ ke saath sigmoid function hai.
\end{itemize}

\section{Ekikrit Suchana Siddhant aur Kathin Samasya se Sambandh}

TC 9.0 Ekikrit Suchana Siddhant (IIT) ka vistaar karta hai, ek pratyaksh ganiteey sambandh sthapit karke:

\begin{equation}
pC = k \cdot \Phi \cdot R(t)
\end{equation}

jahaan $\Phi$ IIT mein paribhashit ekikrit suchana ka pratinidhitv karta hai.

\section{Anubhavjanya Satyapan}
TC 9.0 vartmaan neuroscientific vidhiyon ke saath parikshan yogya vishisht, khandaneey poorvanumaaan utpann karta hai.

\section{Krtrim Buddhimatta ke liye Nikhitarth}

Krtrim pranaliyon ke liye, prabhavi aayatan aur suchana ghanatv ko computational architecture ke sandarbh mein punarsootreekaran ki aavashyakta hai.

\section{Nishkarsh}
TC 9.0 ek ganiteey roop se susangat, anubhavjanya roop se parikshanyogya dhaancha prastut karta hai jo chetana ko seema-seemit property ke roop mein samajhne ke liye hai jo suchana-prasanskaran pranaliyon mein avmandit anunaad ke madhyam se prakat hoti hai. Core equation $pC = k \cdot \rho_I \cdot (1 + A \cdot \sin(\omega t) \cdot e^{-\gamma t})$ with consciousness emergence governed by $C = \sigma(\rho_I - \theta) \cdot pC$ ek ekikrit approach pradaan karta hai jo suchana siddhant, bhautiki, aur neuroscience ko jodata hai.

\end{document}